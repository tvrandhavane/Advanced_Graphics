\documentclass[a4paper,11pt,twocolumn]{article}

\usepackage[dvips]{graphicx}
\usepackage{url}

\title{CS775 Paper Abstract : Locomotion Skills for Simulated Quadrupeds}
\author{Tanmay Randhavane (110050010), Alok Yadav (110050043)}

\date{\today}
\begin{document}
\maketitle

\section{Introduction}
%Use this template to write your paper abstract and produce a pdf using
%pdflatex.

%Introduce the problem solved in the paper in the first section and
%describe why is it interesting.

In this paper~\cite{2011-TOG-quadruped}, the authors have tried to solve the problem of modelling various sets of motions and gaits of a quadruped. Various actions that the authors tried to simulate were simple walk, trot, pace, canter, gallop, jumps over obstacles, falls and recovery after falls, sitting, lying down, getting up.

This problem has direct applications in gaming industry, in robotics (Boston Dynamics's Wildcat ~\cite{Wildcat}) and in movies (The Chronicles of Narnia). Due to these widespread applications, finding a method to create a simulation that is as close to real life as possible becomes necessary.

\section{Problem Scope}
%Talk about the scope of the problem here.
We will try to create a simulation for a dog which can go through various motions and gaits. We will first try to implement the 6 basic gaits including simple walk, trot, pace, canter, transverse gallop and rotary gallop. After modelling this, we will try to optimize these motions to match closely to the data from the filmed dog. After doing this part, if time permits, we will model the other motions of the quadruped like jumps over obstacles, falls and recovery after falls, sitting, lying down and getting up.

We will create the model for dog and the surface in OpenGL. We will code everything else in C++ and use Open Dynamics Engine(ODE) library for forward dynamic simulator.
\section{Solution}

%Summarize the solution here - you can stretch
%your description to maximum to one page
%(both sides of a A4 sheet).


The simulation~\cite{2011-TOG-quadruped-video} is done as follows:
\begin{enumerate}
	\item The motions and gaits of the simulated quadruped (a simulation for dog is used for demonstration) is controlled by various controllers. The controllers include dual leg frames(which are capable of motion independent of each other), a flexible spine (abstracted by joined links) and internal virtual forces.
	\item By changing the aforementioned controllers, torques are computed. These torques are then passed to the forward dynamic simulator at each time step. The authors have considered the forward dynamic simulator as a black box and used the Open Dynamics Engine(ODE) for it.
	\item To set the controllers, gait graphs are used. Gait graphs are plots detecting the swing phase of controllers against the current time. The authors have used experimental data from filmed video of a dog and from data provided by Alexander~\cite{gait_graph} to generate gait graphs for the motions and gaits of the simulated dog.
\end{enumerate}

% \section{Important}
% Learn to use bibtex and how to include citations/references your
% document. Everybody should at least have one reference - of the paper
% you are presenting. You may include more if you refer to them in your
% text.

% The \textbf{MOST} important aspect of writing this abstract is writing
% it in your own words - do NOT copy verbatim from the paper.

% Please write whatever you understand. You are also free to write about
% things in the paper that you think are interesting, difficult, not
% easy to understand or implement. Try to think about what you have
% learnt in graphics so far and place the paper in that context. ~\cite{2011-TOG-quadruped}

\bibliographystyle{abbrv}
\bibliography{references}

\end{document}

