\documentclass[a4paper,11pt,twocolumn]{article}

\usepackage[dvips]{graphicx}

\title{CS775 Paper Abstract : Locomotion Skills for Simulated Quadrupeds}
\author{Tanmay Randhavane (110050010), Alok Yadav (110050043)}

\date{\today}
\begin{document}
\maketitle

\section{Introduction}
%Use this template to write your paper abstract and produce a pdf using
%pdflatex.

%Introduce the problem solved in the paper in the first section and
%describe why is it interesting.

In this paper~\cite{2011-TOG-quadruped}, the authors have tried to solve the problem of modelling various sets of motions and gaits of a quadruped. Various actions that the authors tried to simulate were simple walk, trot, pace, canter, gallop, jumps over obstacles, falls and recovery after falls, sitting, lying down, getting up.

This problem has direct applications in gaming industry, in robotics (Boston Dynamics's Wildcat ~\cite{Wildcat}) and in movies (The Chronicles of Narnia). Due to these widespread applications, finding a method to create a simulation that is as close to real life as possible becomes necessary.

\section{Problem Scope}
%Talk about the scope of the problem here.

\section{Solution}
Summarize the solution here - you can stretch your description to
maximum to one page (both sides of a A4 sheet).

\section{Important}
Learn to use bibtex and how to include citations/references your
document. Everybody should at least have one reference - of the paper
you are presenting. You may include more if you refer to them in your
text.

The \textbf{MOST} important aspect of writing this abstract is writing
it in your own words - do NOT copy verbatim from the paper. 

Please write whatever you understand. You are also free to write about
things in the paper that you think are interesting, difficult, not
easy to understand or implement. Try to think about what you have
learnt in graphics so far and place the paper in that context. ~\cite{2011-TOG-quadruped}

\bibliographystyle{abbrv}
\bibliography{references} 

\end{document}

